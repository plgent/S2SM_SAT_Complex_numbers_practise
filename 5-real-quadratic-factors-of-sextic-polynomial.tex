\question % MASA 2019 Trial Exam, Q5.

\begin{parts}

\part[2]
Use polynomial division to show that $\displaystyle{\frac{z^7 - 1}{z - 1} = z^6 + z^5 + z^4 + z^3 + z^2 + z + 1}$.

\begin{EnvFullwidth}
\begin{solutionorgrid}[2.5in]
On synthetic division we get
\[
    \polyhornerscheme[vars=z, z = 1]{z^7 - 1}
\]
\end{solutionorgrid}
\end{EnvFullwidth}

\part[4]
Hence, for complex $z$, show that the solutions to
\[
	z^6 + z^5 + z^4 + z^3 + z^2 + z^1 + 1 = 0
\]
are $z = \omega^k$, where $\omega = \cis(\frac{2\pi}{7})$ and $1 \leq k \leq 6$ is an integer.
\begin{EnvFullwidth}
\begin{solutionorgrid}[4in]
By part (a) we have
\[
	\frac{z^7 - 1}{z - 1} = z^6 + z^5 + z^4 + z^3 + z^2 + z^1 + 1 \qquad (z \neq 1).
\]
The RHS vanishes if $\displaystyle{\frac{z^7 - 1}{z - 1} = 0}$, which is when
\begin{align*}
    z^7 &= \cis(k2\pi) && (k \in \Z) \\
	z &= \cis(k2\pi)^{\sfrac{1}{7}} \\
	&= \cis\!\pfrac{k2\pi}{7} && (0 \leq k \leq 6).
\end{align*}
The zeroes of $z^6 + z^5 + z^4 + z^3 + z^2 + z^1 + 1$ are $z = \omega^k$, where $\omega = \cis(\frac{2\pi}{7})$ and $1 \leq k \leq 6$ is an integer.
\end{solutionorgrid}
\end{EnvFullwidth}

\part[4]
Find a \emph{real} quadratic factor of $z^6 + z^5 + z^4 + z^3 + z^2 + z^1 + 1$.

\begin{EnvFullwidth}
\begin{solutionorgrid}[4in]
From part (b), $z - \omega$ and $z - \omega^6$ are linear factors of $z^6 + z^5 + z^4 + z^3 + z^2 + z^1 + 1$. Hence, $(z - \omega)(z - \omega^6)$ is a quadratic factor. Note that
\begin{align*}
	\omega^6 &= \cis(\tfrac{2\pi}{7})^6 \\
	&= \cis(\tfrac{12\pi}{7}) \\
	&= \cis(-\tfrac{2\pi}{7}).
\end{align*}
So, $\omega = \cis(\frac{2\pi}{7})$ and $\omega^6 = \cis(-\frac{2\pi}{7})$ are complex conjugates. Their sum is
\begin{align*}
	\cis(\tfrac{2\pi}{7}) + \cis(\tfrac{-2\pi}{7}) &= \cos(\tfrac{2\pi}{7}) + i \sin(\tfrac{2\pi}{7}) + \cos(-\tfrac{2\pi}{7}) + i \sin(-\tfrac{2\pi}{7}) \\
	&= \cos(\tfrac{2\pi}{7}) + i \sin(\tfrac{2\pi}{7}) + \cos(\tfrac{2\pi}{7}) - i \sin(\tfrac{2\pi}{7}) && (\textrm{cosine is even, sine is odd}) \\
	&= 2\cos(\tfrac{2\pi}{7}).
\end{align*}
Their product is
\begin{align*}
	\cis(\tfrac{2\pi}{7}) \times \cis(\tfrac{-2\pi}{7}) &= \cis(0) \\
	&= 1.
\end{align*}
Thus, $z^2 - 2\cos(\frac{2\pi}{7}) + 1$ is a real quadratic factor of $z^6 + z^5 + z^4 + z^3 + z^2 + z^1 + 1$.
\end{solutionorgrid}
\end{EnvFullwidth}

\end{parts}
