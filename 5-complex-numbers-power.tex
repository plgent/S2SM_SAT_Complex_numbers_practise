\question %%% 2011 Midyear, Q5

\begin{parts}

\part[4]
Let $z= 1 + i$ and $w = \sqrt{3} - i$. Write $z$ and $w$ in \emph{exact} $r\cis(\theta)$ form.

\begin{EnvFullwidth}
\begin{solutionorgrid}[2.5in]
We have $\abs{z} = \sqrt{1^2 + 1^2} = \sqrt{2}$ and $\arg(z) = \arctan(1) = \frac{\pi}{4}$. So $z = \sqrt{2} \cis(\frac{\pi}{4})$. Moreover,
\begin{align*}
	\abs{w} &= \sqrt{(\sqrt{3})^2 + (-1)^2} & \arg(w) &= \arctan(\tfrac{-1}{\sqrt{3}}) \\
	&= 2, & &= -\frac{\pi}{6}.
\end{align*}
Thus, $w = 2\cis(-\frac{\pi}{6})$. \emph{Note}: a calculator is a good option for these.
\end{solutionorgrid}
\end{EnvFullwidth}

\part[3]
Find $\displaystyle{\frac{z^4}{w}}$ in the form $r\cis(\theta)$.

\begin{EnvFullwidth}
\begin{solutionorgrid}[1.5in]
We have
\begin{align*}
	\frac{z^4}{w} &= \frac{(\sqrt{2} \cis(\frac{\pi}{4}))^4}{2\cis(-\frac{\pi}{6})}\\
	&= \frac{(2^{\sfrac{1}{2}})^4 \cis(\frac{4\pi}{4})}{2\cis(-\frac{\pi}{6})} \\
	&= 2\cis(\tfrac{12\pi}{12} - (-\tfrac{2\pi}{12})) \\
	&= 2\cis(\tfrac{7\pi}{6}).
\end{align*}
\end{solutionorgrid}
\end{EnvFullwidth}

\part[1]
Hence, use De Moivre's theorem to find the polar form of $\displaystyle{\pfrac{z^4}{w}^{\! n}}$.

\begin{EnvFullwidth}
\begin{solutionorgrid}[1in]
We have
\begin{align*}
	\pfrac{z^4}{w}^{\! n} &= (2\cis(\tfrac{7\pi}{6}))^n \\
	&= 2^n \cis(\tfrac{7n\pi}{6}).
\end{align*}
\end{solutionorgrid}
\end{EnvFullwidth}

\part[2]
Find the smallest positive integer $n$ such that $\displaystyle{\pfrac{z^4}{w}^{\! n}}$ is purely imaginary.

\begin{EnvFullwidth}
\begin{solutionorgrid}[2in]
To be imaginary,
\begin{align*}
	\frac{7n\pi}{6} &= \frac{\pi}{2} + k\pi && (k \in \Z) \\
	7 n\pi &= 3\pi + 6k\pi \\
	7 n &= 3 + 6k.
\end{align*}
If $k = 3$, then $7n = 21$, so $n = 3$.
\end{solutionorgrid}
\end{EnvFullwidth}

\end{parts}
