\question % 2011 Exam, Q12.

\begin{parts}

\part[2]
Show that $\displaystyle{\frac{z^6 + 3z^4 + 3z^2 + 2}{z^2 + 2} = z^4 + z^2 + 1}$.

\begin{EnvFullwidth}
\begin{solutionorgrid}[2.5in]
On division we get
\polyset{vars=z}
\[
	\polylongdiv{z^6 + 3z^4 + 3z^2 + 2}{z^2 + 2}
\]
\end{solutionorgrid}
\end{EnvFullwidth}

\part[3]
Find the purely imaginary zeros of $z^6 + 3z^4 + 3z^2 + 2$.

\begin{EnvFullwidth}
\begin{solutionorgrid}[3.5in]
The first factor in
\[
	z^6 + 3z^4 + 3z^2 + 2 = (z^2 + 2)(z^4 + z^2 + 1).
\]
 vanishes when
\[
	(z + \sqrt{2} i)(z - \sqrt{2} i) = 0.
\]
So, $z = \pm \sqrt{2} i$ are imaginary zeros. For the second factor
\[
	(z^2 + z + 1)(z^2 - z + 1) = 0,
\]
for complex numbers with $\Re(z) \neq 0$.
\end{solutionorgrid}
\end{EnvFullwidth}

\part[3]
Solve $z^3 = -1$ in the form $r\cis(\theta)$.

\begin{EnvFullwidth}
\begin{solutionorgrid}[2in]
We have
\begin{align*}
	z^3 &= \cis(\pi + k2\pi) && (k \in \Z) \\
	z &= \cis(\pi + k2\pi)^{\sfrac{1}{3}} \\
	&= \cis\! \pfrac{\pi + k2\pi}{3}.
\end{align*}
\end{solutionorgrid}
\end{EnvFullwidth}

\part[4]
Hence, solve $(z^2 + 1)^3 = -1$, showing clearly that
\[
	z = \pm \sqrt{2} i, \, z = \cis(\tfrac{\pi}{3}), \, z = \cis(\tfrac{2\pi}{3}), \, z = \cis(-\tfrac{2\pi}{3}), \, z = \cis(-\tfrac{\pi}{3}).
\]

\begin{EnvFullwidth}
\begin{solutionorgrid}[4in]
By taking the cube root of both sides
\begin{align*}
	z^2 + 1 &= (-1)^{\sfrac{1}{3}} \\
	&= \cis\! \pfrac{\pi + k2\pi}{3}.
\end{align*}
So,
\[
	z^2 + 1 = -1, \qquad z^2 + 1 = \cis(\tfrac{\pi}{3}), \qquad z^2 + 1 = \cis(\tfrac{-\pi}{3}).
\]
The first equation is $z^2 + 2 = 0$, whose roots are $z = \pm \sqrt{2} i$. The others are
\begin{align*}
	z^2 &= \cis(\tfrac{\pi}{3}) - 1 & z^2 &= \cis(-\tfrac{\pi}{3}) - 1 \\
	&= \cis(\tfrac{2\pi}{3}) & &= \cis(\tfrac{4\pi}{3}) \\
	z &= \pm \cis(\tfrac{\pi}{3}), & z &= \pm \cis(\tfrac{2\pi}{3}).
\end{align*}
Thus, these solutions are
\[
	z = \cis(\tfrac{\pi}{3}), \, z = \cis(\tfrac{2\pi}{3}), \, z = \cis(-\tfrac{2\pi}{3}), \, z = \cis(-\tfrac{\pi}{3}).
\]
\end{solutionorgrid}
\end{EnvFullwidth}

\uplevel{Let the solutions to $(z^2 + 1)^3 = -1$ be $z_1$, $z_2$, $z_3$, $z_4$, $z_5$ and $z_6$, with $z_1$ in the first quadrant and arguments increasing anticlockwise from the positive real axis.}

\part[2]
Graph and label the solutions on the Argand diagram in Figure~\ref{fig:plotting-on-the-argand-diagram}.

\begin{figure}[h]
	\centering
	\begin{tikzpicture}	
	\begin{axis}[
		my axis style,
		axis equal image,
		height=3.5in,
		xmin=-2,
		xmax=2,
		ymin=-2,
		ymax=2,
		minor tick num=1,
		grid=both,
		xlabel=$\Re$,
		ylabel=$\Im$,
	]
	\coordinate (O) at (axis cs:0,0);
	\coordinate (Z1) at (axis cs:.5,.866);
	\coordinate (Z2) at (axis cs:0,1.414);
	\coordinate (Z3) at (axis cs:-.5,.866);
	\coordinate (Z4) at (axis cs:-.5,-.866);
	\coordinate (Z5) at (axis cs:0,-1.414);
	\coordinate (Z6) at (axis cs:.5,-.866);
	\ifprintanswers
	\fill[red] (O);
	\addplot[gray, dashed] ({cos(x)}, {sin(x)});
	\addplot[gray, dashed] ({sqrt(2)*cos(x)}, {sqrt(2)*sin(x)});
	\draw[
	    vector,
	    red,
	] (O) -- (Z1);
	\draw[
		vector,
		red,
	] (O) -- (Z6);
	\draw[
	    vector,
	    ultra thick,
	    red,
	] (Z6) -- (Z1);
	\fill[red] (O) circle (2pt) (Z1) circle (2pt) node[above right] {$z_1$} (Z2) circle (2pt) node[above right] {$z_2$} (Z3) circle (2pt) node[above left] {$z_3$} (Z4) circle (2pt) node[below right] {$z_4$} (Z5) circle (2pt) node[below right] {$z_5$} (Z6) circle (2pt) node[below right] {$z_6$};
	\else
	\fi
	\end{axis}

%	\ifprintanswers
%	\fill[
%		red
%	]
%	(Z1) circle (2pt) node[above right] {$z_1$}
%		(Z2) circle (2pt) node[above right] {$z_2$}
%		(Z3) circle (2pt) node[above left] {$z_3$}
%		(Z4) circle (2pt) node[below right] {$z_4$}
%		(Z5) circle (2pt) node[below right] {$z_5$}
%		(Z6) circle (2pt) node[below right] {$z_6$}
%	;
%	\else
%	\fi
	
	\end{tikzpicture}
	\caption{Draw the six distinct solutions here.}
	\label{fig:plotting-on-the-argand-diagram}
\end{figure}

\part
Using your diagram:

\begin{subparts}

\subpart[2]
Show that $\abs{z_1 - z_6} = \sqrt{3}$.

\begin{EnvFullwidth}
\begin{solutionorgrid}[2in]
The vector $z_1 - z_6$ has length
\begin{align*}
	\abs{z_1 - z_6} &= 2 \times \frac{\sqrt{3}}{2} \\
	&= \sqrt{3}.
\end{align*}
\end{solutionorgrid}
\end{EnvFullwidth}

\subpart[2]
Find $\abs{z_1 - z_6} + \abs{z_2 - z_5} + \abs{z_3 - z_4}$.

\begin{EnvFullwidth}
\begin{solutionorgrid}[2.5in]
We have $\abs{z_2 - z_5} = 2 \sqrt{2}$ and $\abs{z_3 - z_4} = \sqrt{3}$. Thus,
\begin{align*}
	\abs{z_1 - z_6} + \abs{z_2 - z_5} + \abs{z_3 - z_4} &= 2\sqrt{2} + 2\sqrt{3} \\
	&= 2(\sqrt{2} + \sqrt{3}).
\end{align*}
\end{solutionorgrid}
\end{EnvFullwidth}

\end{subparts}

\end{parts}
