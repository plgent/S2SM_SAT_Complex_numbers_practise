\question % 2012 Exam, Q5.
Let $P(x) = x^3 + ax + b$, where $a$ and $b$ are constants.

\begin{parts}

\part[2]
Given that $x + 1$ is a factor of $P(x)$, show that $b - a = 1$.

\begin{EnvFullwidth}
\begin{solutionorgrid}[1.75in]
By the factor theorem, $P(-1) = 0$ but
\begin{align*}
	P(-1) &= (-1)^3 - a + b \\
	&= b - a - 1
\end{align*}
and
\begin{align}
    b - a - 1 &= 0 \nonumber \\
    b - a &= 1 \label{eq:1}.
\end{align}
\end{solutionorgrid}
\end{EnvFullwidth}

\part[1]
When $P(x)$ is divided by $x + 2$, the remainder is $6$. Show that $b - 2a = 14$.

\begin{EnvFullwidth}
\begin{solutionorgrid}[1.75in]
By the remainder theorem, $P(-2) = 6$ but
\begin{align*}
	P(-2) &= (-2)^3 - 2a + b \\
	&= b - 2a - 8
\end{align*}
and
\begin{align}
    b - 2a - 8 &= 6 \nonumber \\
	b - 2a &= 14 \label{eq:2}.
\end{align}
\end{solutionorgrid}
\end{EnvFullwidth}

\part[2]
Find $a$ and $b$, and hence write $P(x)$ as a product of linear factors.

\begin{EnvFullwidth}
\begin{solutionorgrid}[2in]
Subtracting (\ref{eq:1}) from (\ref{eq:2}) leaves $-a = 13$, so $a = -13$. Thus, $b = -12$ and
\begin{align*}
	P(x) &= x^3 - 13x - 12 \\
	&= (x + 1)(x^2 - x - 12) \\
	&= (x + 3)(x + 1)(x - 4).
\end{align*}
\end{solutionorgrid}
\end{EnvFullwidth}

\part[2]
Divide $P(x)$ from part (c) by $x - 4$.

\begin{EnvFullwidth}
\begin{solutionorgrid}[1.5in]
Through synthetic division we get
\[
    \polyhornerscheme[x = 4]{x^3 - 13x - 12}
\]
\end{solutionorgrid}
\end{EnvFullwidth}

\end{parts}
